%% Document type
\documentclass[]{amsbook}

%% Packages
\usepackage{amsmath}
\usepackage{amsfonts}
\usepackage{amssymb}
\usepackage{exercise}
\usepackage{hyperref}

%% User-defined commands
\newcommand{\q}{\quad}
\newcommand{\qq}{\qquad}
\newcommand{\catname}[1]{\mathbf{#1}}
\newcommand{\N}{\mathbb{N}}
\newcommand{\Z}{\mathbb{Z}}
\newcommand{\Q}{\mathbb{Q}}
\newcommand{\R}{\mathbb{R}}
\newcommand{\0}{\mathbf{0}}
\newcommand{\1}{\mathbf{1}}
\newcommand{\oldemptyset}{\emptyset}
\renewcommand{\emptyset}{\varnothing}

\newtheorem{prop}{Proposition}
\newenvironment{solution}
    {\begin{proof}[Solution]}{\end{proof}}


\begin{document}
\title{Introduction to Categories and Categorical Logic}
\author{Vishal Lama}
\date{\today}
\maketitle
\tableofcontents

\chapter{Introduction to Categories and Categorical Logic}
\section{Introduction}
We say that a function $f: X \to Y$ is:\\~\\
\begin{tabular}{ l l }
	\emph{injective} & if $\forall x, x' \in X. f(x) = f(x') \implies x = x'$, \\
	\emph{surjective} & if $\forall y \in Y. \exists x \in X. f(x) = y$, \\~\\
	\emph{monic} & if $\forall g, h. f \circ g = f \circ h \implies g = h$
	\q ($f$ is left cancellative),\\
	\emph{epic} & if $\forall g, h. g \circ f = h \circ f \implies g = h$
	\q ($f$ is right cancellative).
\end{tabular}\\

\begin{prop}
	Let $f : X \to Y$. Then,
	\begin{enumerate}
		\item $f$ is injective $\iff f$ is monic.
		\item $f$ is surjective $\iff f$ is epic.
	\end{enumerate}
\end{prop}
\begin{proof}
	We first show $(1)$.\\
	($\impliedby$) Suppose $f$ is monic. Fix a one-element set
	$\boldsymbol{1} = \{ \bullet \}$. Then, note that elements $x \in X$ are in
	1-1 correspondence with functions $\bar{x}: \boldsymbol{1} \to X$, defined
	by $\bar{x}(\bullet) := x$. Then, for all $x, x' \in X$, we have\\~\\
	\begin{tabular} { l l }
		& $f(x) = f(x')$\\
		$\implies$ & $f(\bar{x}(\bullet)) = f(\bar{x'}(\bullet))$\\
		$\implies$ & $(f \circ \bar{x})(\bullet) = (f \circ \bar{x'})(\bullet)$\\
		$\implies$ & $f \circ \bar{x} = f \circ \bar{x'}$\\
		$\implies$ & $\bar{x} = \bar{x'}$ \q (since $f$ is monic) \\
		$\implies$ & $\bar{x}(\bullet) = \bar{x'}(\bullet)$\\
		$\implies$ & $x = x'$
	\end{tabular}\\
	This shows that $f$ is injective.\\

	($\implies$) Suppose $f$ is injective. Let $f \circ g = f \circ h$ for all
	$g, h: A \to X$. Then, for all $a \in A$,\\
	\begin{tabular} { l l }
		& $(f \circ g)(a) = (f \circ h)(a)$\\
		$\implies$ & $f(g(a)) = f(h(a))$\\
		$\implies$ & $g(a) = h(a)$ \q (since $f$ is injective)\\
		$\implies$ & $g = h$
	\end{tabular}\\
	This establishes that $f$ is monic. And, we are done.
\end{proof}

\setcounter{Exercise}{1}
\begin{Exercise}
	Show that $f: X \to Y$ is surjective iff it is epic.
\end{Exercise}
\begin{solution}
	($\implies$) Suppose $f: X \to Y$ is epic. And, assume, for the sake of contradiction,
	$f$ is \emph{not} surjective. Then, there exists some $y_0 \in Y$, such
	that, for all $x \in X$, $f(x) \ne y_0$. Define mappings $g, h: Y \to Y \cup \{ Y \}$
	by:
	\begin{center}
		$g(y) := y$\\~\\
		$h(y) :=
		\begin{cases}
		y & \text{if } y \ne y_0 \\
		Y & \text{if } y = y_0
		\end{cases}$
	\end{center}
	Note that $g \ne h$.\\
	Then, for all $x \in X$, $(g \circ f)(x) = g(f(x)) = h(f(x)) = (h \circ f)(x)$.
	This implies $g \circ f = h \circ f$, which implies $g = h$, since $f$ is
	epic. The last conclusion contradicts the fact that $g = h$. Thus, we
	conclude $f$ is	surjective.\\~\\
	($\impliedby$) Suppose $f: X \to Y$ is surjective. Then, for any $y \in Y$,
	there exists an $x \in X$, such that $f(x) = y$. Now, assume, for all
	$g, h: Y \to Z$, $g \circ f = h \circ f$. Then, for all $y \in Y$,
	$g(y) = g(f(x)) = (g \circ f)(x) = (h \circ f)(x) = h(f(x)) = h(y)$, which
	implies $g = h$, showing that $f$ is epic.\\
	And, this completes our proof.
\end{solution}

\setcounter{Exercise}{4}
\begin{Exercise}
	Suppose $G$ and $H$ are groups (and hence monoids), and that $h: G \to H$
	is a monoid homomorphism. Prove that $h$ is a group homomorphism.
\end{Exercise}
\begin{solution}
	We need only show that $h$ preserves inverses. To that end,
	suppose $g^{-1}$ is the inverse of $g \in G$. Then, $h(g) h(g^{-1}) =
	h(g g^{-1}) = h(1_G) = 1_H = h(1_G) = h(g^{-1} g) = h(g^{-1}) h(g)$.
	This establishes $h$ preserves inverses, and we are done.
\end{solution}

\begin{Exercise}
	Check that $\catname{Mon}, \catname{Vect}_k, \catname{Pos}$, and
	$\catname{Top}$ are indeed categories.
\end{Exercise}
\begin{solution}
	($\catname{Mon}$) The objects are monoids $(M, \cdot, 1_M)$, and morphisms
	are monoid homomorphisms. Given monoid homomorphisms, $f: (M, \cdot, 1_M)
	\to (N, \cdot, 1_N)$ and $g: (N, \cdot, 1_N) \to (P, \cdot, 1_P)$, the
	function $g \circ f: (M, \cdot, 1_M) \to (P, \cdot, 1_P)$ is also a monoid
	homomorphism, because for all $m, m' \in M$, we have $(g \circ f)(m m') =
	g(f(m m')) = g(f(m) f(m')) = (g(f(m)) (g(f(m'))) = ((g \circ f)(m))
	((g \circ f)(m'))$. Also, for each monoid, the identity morphism is the
	identity function. It is also easy to check that for all monoid
	homomorphisms $f, g$ and $h$ with the appropriate domains and codomains,
	$h \circ (g \circ f) = (h \circ g) \circ f$. This establishes that
	$\catname{Mon}$ is indeed a category.

	($\catname{Vect}_k$) The objects are vector spaces over a field $k$, and
	morphisms are linear maps between vector spaces. Suppose $f: U \to V$ and
	$g: V \to W$ are linear maps. Then, for all $x, y \in U$, we have
	$(g \circ f)(x + y) = g(f(x + y)) = g(f(x) + f(y)) = g(f(x)) + g (f(y)) =
	(g \circ f)(x) + (g \circ f)(y)$. Also, for all $\alpha \in k$, we have
	$(g \circ f)(\alpha x) = g(f(\alpha x)) = g(\alpha f(x)) = \alpha g(f(x))
	= \alpha (g \circ f)(x)$. This establishes $g \circ f: U \to W$ is a
	linear map as well. The identity map $1_U$ for any vector space $U$ is the
	identity morphism. The associativity of linear maps and the identity axiom
	follow from the	property of functions. This shows that $\catname{Vect}_k$
	is also a category.

    ($\catname{Pos}$) The objects are partially ordered sets, and morphisms
    are monotone functions between these sets. Suppose $h: P \to Q$ and
    $g: Q \to R$ are monotone functions. Then, for all $x, y \in P$,
    $x \le y \implies h(x) \le h(y) \implies g(h(x)) \le g(h(y)) \implies
    (g \circ h)(x) \le (g \circ h)(y)$, which shows $g \circ h: P \to R$ is
    a monotone function. The identity map is the identity morphism, and the
    associativity and identity axioms are satisfied by the property of
    functions. This establishes $\catname{Pos}$ is a category.

    ($\catname{Top}$) The objects are topological spaces, and morphisms are
    continuous maps between these spaces. Given continuous maps $f: (X, T_X)
    \to (Y, T_Y)$ and $g: (Y, T_Y) \to (Z, T_Z)$, we can show that $g \circ f:
    (X, T_X) \to (Z, T_Z)$ is also a continuous map. First, note that for any
    $T \subset Z$, $x \in (g \circ f)^{-1}(T)$ iff $(g \circ f)(x) \in T$ iff
    $g(f(x)) \in T$ iff $f(x) \in g^{-1}(T)$ iff $x \in f^{-1}(g^{-1}(T))$.
    Thus,
    \begin{center}
	    for all $T \subset Z, (g \circ f)^{-1}(T) = f^{-1}(g^{-1}(T))$.
    \end{center}
	Therefore, for any open set $T \in T_Z$, we have $g^{-1}(T) \in T_Y$,which
	implies	$f^{-1}(g^{-1}(T)) \in T_X$, which implies $(g \circ f)^{-1}(T)
	\in T_X$ (by using the result above.) Hence, $g \circ f: (X, T_X) \to
	(Z, T_Z)$ is a continuous map. The associativity and identity axioms follow
	from the associativity and identity laws for functions. This establishes
	$\catname{Top}$ is a category.
\end{solution}

\begin{Exercise}
    Check carefully that monoids correspond exactly to one-object categories.
    Make sure you understand the difference between such a category and
    $\catname{Mon}$. (For example: how many objects does $\catname{Mon})$
    have?)
\end{Exercise}
\begin{solution}
    (\href{https://ncatlab.org/nlab/show/monoid#as_a_oneobject_category}
    {Monoid as a one-object category}) Given a monoid $(M, \cdot, 1)$,
    we can construct its corresponding category as follows. We write
    $\catname{B}M$ for the corresponding category with a single object
    $\bullet$, where $\catname{Hom}_{\catname{B}M}(\bullet, \bullet) := M$.
    We note then that the composition map in $\catname{B}M$ is reflected in
    the binary operation $\_ \cdot \_ : M \times M \to M$, where
    $\mathbf{id}_{\bullet} := 1$. Then, the associative and identity laws for
    the category $\catname{B}M$ follow directly from the associative and
    identity laws, respectively, satisfied by the monoid $(M, \cdot, 1)$.
    This shows any monoid can be seen or interpreted as a one-object category.
\end{solution}

\begin{Exercise}
    Check carefully that preorders correspond exactly to categories in which
    each homset has at most one element. Make sure you understand the
    difference between such a category and $\catname{Pos}$. (For example: how
    big can homsets in $\catname{Pos}$ be?)
\end{Exercise}
\begin{solution}
	Let $(P, \le)$ be a preorder. Then, we define the corresponding category
	$\catname{C}$ as follows. The objects of $\catname{C}$ are the elements of
	the set $P$, and for all $x, y \in P$, we define a morphism $x \to y$ iff
	$x \le y$. Then, for every object $x \in \catname{C}$, the identity
	morphism $1_x: x \to x$ corresponds exactly to the reflexive property
	$x \le x$ for all $x \in P$. Note that each homset in $\catname{C}$ has at
	most one element. Also, for every $x \to y$ and $y \to z$ in $\catname{C}$,
	$x \to z$ follows from the fact that $x \le y$ and $y \le z$ and the
	transitivity of the $\le$ relation on $P$. This defines a composition map
	for morphisms in $\catname{C}$. In addition, for all morphisms $x \to y$,
	$y \to z$, and $z \to w$, their associativity follows immediately from the
	transitivity of $\le$. Lastly, the unit laws also follow from the same
	transitivity relation. Therefore, we conclude that every preorder
	corresponds precisely to a category in which each homset has at most one
	element.
\end{solution}

\setcounter{Exercise}{9}
\begin{Exercise}
	Show that the inverse, if it exists, is unique.
\end{Exercise}
\begin{solution}
	Suppose $i: A \to B$ is an isomorphism, with inverse $j: B \to A$, in a
	category $\catname{C}$. Suppose $j': B \to A$ is also an inverse of $i$.
	Then, $j = 1_A \circ j = (j' \circ i) \circ j = j' \circ (i \circ j) =
	j' \circ 1_B = j'$, and we are done.
\end{solution}

\begin{Exercise}
	Show that $\cong$ is an equivalence relation on the objects of a category.
\end{Exercise}
\begin{solution}
	Let $\catname{C}$ be some category.\\
	(\emph{Reflexivity}) For any object $X \in \catname{C}$, $X \cong X$
	follows from the fact that the identity morphism $1_X: X \to X$ is an
	isomorphism.\\
	(\emph{Symmetry}) If $X \cong Y$, then there exists an isomorphism $i: X
	\to Y$. But, the inverse, $i^{-1}: Y \to X$, of $i$ is also an isomorphism.
	Hence, $Y \cong X$.\\
	(\emph{Transitivity}) Suppose $X \cong Y$ and $Y \cong Z$. Then, there
	exist isomorphisms $i: X \to Y$ and $j: Y \to Z$. Then, we claim that
	$j \circ i: X \to Z$ is also an isomorphism. Indeed, its trivial to show
	that its inverse is the morphism $i^{-1} \circ j^{-1}: Z \to X$. This
	implies $X \cong Z$.\\
	We thus conclude that $\cong$ is an equivalence relation on the objects of
	a category.
\end{solution}

\begin{Exercise}
	Verify the claims that isomorphisms in $\catname{Set}$ correspond exactly
	to bijections, in $\catname{Grp}$ to group isomorphisms, in $\catname{Top}$
	to homeomorphisms, and in $\catname{Pos}$ to isomorphisms.
\end{Exercise}
\begin{solution}
    ($\catname{Set}$) We claim the following:
    \begin{enumerate}
        \item $f: X \to Y$ is injective iff $f$ has a left inverse.
        \item $f: X \to Y$ is surjective iff $f$ has a right inverse.
    \end{enumerate}
    We first show (1).

    ($\implies$) Suppose $f: X \to Y$ has a left inverse, $g: Y \to X$, say.
    Then, $g \circ f = 1_X$. Assume for any $x, x' \in X, f(x) = f(x')$. Then,
    $x = 1_X(x) = (g \circ f)(x) = g(f(x)) = g(f(x')) = (g \circ f)(x') =
    1_X(x') = x'$, which implies $f$ is injective.

    ($\impliedby$) Suppose $f: X \to Y$ is injective. If $X$ is empty, then $f$
    is an empty function corresponding to each $Y$. In this case, $1_X$ is also
    an empty function, and we thus have $g \circ f = 1_X$ for any $g: Y \to X$.
    That is, $f$ has a left inverse. On the other hand, if $X$ is nonempty,
    choose some $x_0 \in X$. Define $g: Y \to X$ by
    \begin{center}
        $g(y) :=
        \begin{cases}
        x_0 & \text{if } y \in Y \setminus \mathbf{Im}(f) \\
        f^{-1}(y) & \text{if } y \in \mathbf{Im}(f)
        \end{cases}$
    \end{center}
    Then, for all $x \in X, (g \circ f)(x) = g(f(x)) = x = 1_X(x)$, which
    implies $g \circ f = 1_X$, thus showing that $g$ is a left inverse of $f$.

    We now show (2).

    ($\implies$) Suppose $f: X \to Y$ has a right inverse, $g: Y \to X$, say.
    Then, $f \circ g = 1_Y$. Therefore, for all $y \in Y$, $y = 1_Y(y) =
    (f \circ g)(y) = f(g(y)) = f(x)$, where $x = g(y)$. This shows $f$ is
    surjective.

    ($\impliedby$) Suppose $f: X \to Y$ is surjective. Now, consider an indexed
    family of nonempty sets $\{ f^{-1}(y)\}_{y \in Y}$. Then, using the axiom
    of choice, we conclude there exists a function $g: Y \to X$, such that $g(y)
    \in f^{-1}(y)$ for all $y \in Y$. Then, for all $y \in Y$, $(f \circ g)(y) =
    f(g(y)) = y = 1_Y(y)$, which implies $f \circ g = 1_Y$, thus proving $f$ has
    a right inverse.

    Since in $\catname{Set}$ a bijection is a function which is both injective
    and surjective, using (1) and (2), we immediately conclude that bijections
    in $\catname{Set}$ correspond exactly to isomorphisms, and we are done.

    In addition, in any category $\catname{C}$, if $f: X \to Y$ has both a left
    inverse, $g: Y \to X$, say, and a right inverse, $h: Y \to X$, say, then
    $g = h$. Indeed, $g = g \circ 1_Y = g \circ (f \circ h) = (g \circ f) \circ
    h = 1_X \circ h = h$, and we are done.\\

    % TODO
    ($\catname{Grp}$)

    % TODO
    ($\catname{Top}$)

    % TODO
    ($\catname{Pos}$)
\end{solution}

\subsection*{Opposite Categories and Duality}
Given a category $\catname{C}$, the opposite category $\catname{C}^{\mathbf{op}}$
is given by taking the same objects as $\catname{C}$, and
\begin{center}
    $\catname{C}^{\mathbf{op}}(A, B) = \catname{C}(B, A)$.
\end{center}

Composition and identities are inherited from $\catname{C}$.\\
If we have
\begin{center}
    $A \xrightarrow{f} B \xrightarrow{g} C$
\end{center}
in $\catname{C}^{\mathbf{op}}$, this means
\begin{center}
    $A \xleftarrow{f} B \xleftarrow{g} C$
\end{center}
in $\catname{C}$. Therefore, composition $ g \circ f$ is
$\catname{C}^{\mathbf{op}}$ is defined as $f \circ g$ in $\catname{C}$. This
leads to the \emph{\textbf{principle of duality}}: a statement $S$ is true
about a category $\catname{C}$ iff its dual (\emph{i.e.} the one obtained
from $S$ by reversing all the arrows) is true about $\catname{C}^{\mathbf{op}}$.
For example, a morphism $f$ is monic in $\catname{C}^{\mathbf{op}}$ iff it is
epic in $\catname{C}$. We say monic and epic are \emph{dual notions}.

\setcounter{Exercise}{13}
\begin{Exercise}
    If $P$ is a preorder, for example $(\R, \le)$, describe $P^{\mathbf{op}}$
    explicitly.
\end{Exercise}
\begin{solution}
    An arrow $a \le_{P^{\mathbf{op}}} b$ in $P^{\mathbf{op}}$ is precisely the
    arrow $b \le_P a$ in $P$. When $P = (\R, \le)$, $P^{\mathbf{op}}$ describes
    the ``greater than or equal" preorder relation on $\R$.
\end{solution}

\subsection*{Subcategories}
Let $\catname{C}$ be a category. Suppose we are given the collections
\begin{center}
    $\mathbf{Ob}(\catname{D}) \subseteq \mathbf{Ob}(\catname{C})$,\\
    $\forall A, B \in \mathbf{Ob}(\catname{D}).
    \catname{D}(A, B) \subseteq \catname{C}(A, B)$.
\end{center}
We say $\catname{D}$ is a \emph{\textbf{subcategory}} of $\catname{C}$ if
it is itself a category. In particular, $\catname{D}$ is:
\begin{itemize}
    \item A \emph{\textbf{full}} subcategory of $\catname{C}$ if for any
    $A, B \in \mathbf{Ob}(\catname{D})$, $\catname{D}(A, B) = \catname{C}(A, B)$.
    \item  A \emph{\textbf{lluf}} subcategory of $\catname{C}$ if
    $\mathbf{Ob}(\catname{D}) = \mathbf{Ob}(\catname{C})$.
\end{itemize}
For example, $\catname{Grp}$ is a full subcategory of $\catname{Mon}$, and
$\catname{Set}$ is a lluf subcategory of $\catname{Rel}$.

\setcounter{Exercise}{15}
\begin{Exercise}
    How many categories $\catname{C}$ with $\mathbf{Ob}(\catname{C}) =
    \{ \bullet \}$ are there? (Hint: what do such categories correspond to?)
\end{Exercise}
\begin{solution}
    Each such category corresponds to a monoid. So, there are as many such
    categories as there are monoids.
\end{solution}

\subsection*{Exercises}
\begin{enumerate}
    \item Consider the following properties of an arrow $f$ in a category
    $\catname{C}$.
    \begin{itemize}
        \item $f$ is \emph{split monic} if for some $g$, $g \circ f$ is an
        identity arrow.
        \item $f$ is \emph{split epic} if for some $g$, $f \circ g$ is an
        identity arrow.
    \end{itemize}
    \begin{itemize}
        \item[a.] Prove that if $f$ and $g$ are arrows such that $g \circ f$ is
        monic, then $f$ is monic.
        \item[b.] Prove that if $f$ is split epic then it is epic.
        \item[c.] Prove that if $f$ and $g \circ f$ are iso then $g$ is iso.
        \item[d.] Prove that if $f$ is monic and split epic then it is iso.
        \item[e.] In the category $\catname{Mon}$ of monoids and monoid
        homomorphisms, consider the inclusion map
        \begin{center}
            $i: (\N, +, 0) \to (\Z, +, 0)$
        \end{center}
        of natural numbers into the integers. Show that this arrow is both monic
        and epic. Is it an iso?
    \end{itemize}
    The \textbf{Axiom of Choice} in Set Theory states that if
    $\{ X_i \}_{i \in I}$ is a family of nonempty sets, we can form  a set
    $X = \{ x_i \mid i \in I \}$, where $x_i \in X_i$ for all $i \in I$.
    \begin{itemize}
        \item[f.] Show that in $\catname{Set}$ an arrow which is epic is split
        epic. Explain why this needs the Axiom of Choice.
        \item[g.] Is is always the case that an arrow which is epic is split
        epic? Either prove that it is, or give a counterexample.
    \end{itemize}

    \item Give a description of partial orders as categories of a special kind.
\end{enumerate}
\begin{solution}
    \leavevmode
    \begin{enumerate}
        \item \leavevmode
        \begin{itemize}
            \item[a.] Suppose $f: A \to B$ and $g: B \to C$ such that
            $g \circ f$ is monic. Assume, for all $i, j: Z \to A$,
            $f \circ i = f \circ j$. Then, $(g \circ f) \circ i = g \circ
            (f \circ i) = g \circ (f \circ j) = (g \circ f) \circ j$, which
            implies $i = j$, since $g \circ f$ is monic. This implies $f$ is
            monic, and we are done.
            \item[b.] Suppose $f: A \to B$ is split epic. Then, there exists a
            $g: B \to A$ such that $f \circ g = 1_B$. Assume, for all $i, j: B
            \to C$, $i \circ f = j \circ f$. Then, $i = i \circ 1_B = i \circ
            (f \circ g) = (i \circ f) \circ g = (j \circ f) \circ g = j \circ
            (f \circ g) = j \circ 1_B = j$, which shows $f$ is epic.
            \item[c.] Suppose $f: A \to B$ and $g: B \to C$ such that $f$ and
            $g \circ f$ are iso. We claim that the inverse of $g$ is
            $f \circ (g \circ f)^{-1}: C \to B$. Indeed, $g \circ (f \circ
            (g \circ f)^{-1}) = (g \circ f) \circ (g \circ f)^{-1} = 1_C$, and
            $(f \circ (g \circ f)^{-1}) \circ g = f \circ (g \circ f)^{-1} \circ
            (g \circ f) \circ f^{-1} = f \circ f^{-1} = 1_B$, which establishes
            $g$ is also an iso.
            \item[d.] Suppose $f: A \to B$ is monic and split epic. The latter
            implies $f$ has a right inverse, $g: B \to A$, say, where $f \circ g
            = 1_B$. Note that $g \circ f: A \to A$ and $1_A: A \to A$. Now,
            $f \circ (g \circ f) = (f \circ g) \circ f = 1_B \circ f = f = f
            \circ 1_A$, which implies $g \circ f = 1_A$, since $f$ is monic
            (left cancellative). Thus, $g$ is also a left inverse of $f$, and
            hence, $f$ is iso.
            \item[e.] It is easy to prove the inclusion map $\N \hookrightarrow
            \Z$ is really a monoid homomorphism. Indeed, $i(0) = 0$, and, for all
            $n_1, n_2 \in \N$, $i(n_1 + n_2) = n_1 + n_2 = i(n_1) + i(n_2)$.\\
            Next, we show that $i$ is monic. Assume, for all monoid homomorphisms
            $g, h: X \to \N$, $i \circ g = i \circ h$. Then, for all $x \in X$,
            $(i \circ g)(x) = (i \circ h) (x)$, which implies $i(g(x)) = i(h(x))$,
            which implies $g(x) = h(x)$, which implies $g = h$. This shows the
            inclusion map is monic.\\
            We now show the inclusion map is epic. First, assume, for all
            monoid homomorphisms $g, h: (\Z, +, 0) \to (X, \star, 1_X)$,
            $g \circ i = h \circ i$. Then, for all $n \in \N$, $(g \circ i)(n) = (h \circ i)(n)$, which
            implies $g(i(n)) = h(i(n))$, which implies $g(n) = h(n)$. We now
            claim that for all $n \ge 1$, $g(-n) = h(-n)$. To that end, we use
            induction on $n$. Note that $g(-1) = g(-1) \star 1_X = g(-1) \star
            h(0) = g(-1) \star h(1 + (-1)) = g(-1) \star h(1) \star h(-1) = g(-1)
            \star g(1) * h(-1) = g(-1 + 1) \star h(-1) = g(0) \star h(-1) = 1_X
            \star h(-1) = h(-1)$. Now, assume the proposition holds for some
            $n \ge 1$. Then, $g(-(n + 1)) = g(-n + (-1)) = g(-n) \star g(-1) =
            h(-n) \star h(-1) = h(-n + (-1)) = h(-(n + 1))$. Hence, by induction,
            $g(-n) = h(-n)$ for all $n \ge 1$. Combining the results from above,
            we thus conclude $g(z) = h(z)$ for all $z \in \Z$. In other words,
            $g = h$, which implies $i$ is epic.\\
            Clearly, the inclusion map $\N \hookrightarrow \Z$ is not iso.
            \item[f.] Suppose $f: X \to Y$ is epic in $\catname{Set}$. Then,
            from an earlier result about $\catname{Set}$, we conclude $f$ is
            surjective. Now, consider the family of nonempty sets
            $\{ f^{-1}(b) \}_{b \in B}$. Each of the sets in the family is
            nonempty, because $f$ is surjective. Therefore, using the Axiom of
            Choice, we can choose some element from each nonempty set in the
            family to construct a function $g: Y \to X$, given by $g(y) := x$
            if $x \in f^{-1}(b)$. In addition, for all $y \in Y$, $(f \circ g)(y)
            = f(g(y)) = y = 1_Y(y)$, which implies $f \circ g = 1_Y$. This shows
            $f$ has a right inverse, thus proving $f$ is split epic.
            \item[g.] It isn't always the case that an arrow which is epic is
            split epic. For example, in the category $\catname{Mon}$, the
            inclusion map $\N \hookrightarrow \Z$ is epic (as shown in (e)
            above.) Now, if we assume that it is also split epic, then there
            exists a monoid homomorphism $g: \Z \to \N$, such that $i \circ g =
            1_{\Z}$. This implies $(i \circ g)(-1) = 1_{\Z}(-1)$, which implies
            $i(g(-1)) = -1$, which implies $g(-1) = -1$, which implies $-1 \in
            \N$, which is absurd. We thus conclude the aforesaid inclusion map
            is \emph{not} split epic, even though it is epic. And this proves
            our original claim.
        \end{itemize}
    \item Suppose $(P, \le)$ is a poset. Then, its corresponding category
    $\catname{C}$ is defined as follows. The objects of $\catname{C}$ are the
    elements of $P$, and for all $x, y \in P$, $x \to y$ iff $x \le y$. The
    reflexivity of $\le$ corresponds to the identity arrows, and transitivity
    to arrow composition. Note that there is at most one arrow for every pair
    of objects in the category. Anti-symmetry of $\le$ corresponds to the fact
    that the only isomorphisms in $\catname{C}$ are the identity arrows.
    \end{enumerate}
\end{solution}

\section{Some Basic Constructions}
An object $I$ in a category $\catname{C}$ is \emph{\textbf{initial}} if, for
every object $A$, \emph{there exists a unique arrow} $I \to A$, which we write
$\iota_{A}: I \to A$.

An object $T$ in a category $\catname{C}$ is \emph{\textbf{terminal}} if, for
every object $A$, \emph{there exists a unique arrow} $A \to T$, which we write
$\tau_{A}: A \to T$.

Note that initial and terminal objects are dual notions: $T$ is terminal in
$\catname{C}$ iff it is initial in $\catname{C}^{\mathbf{op}}$. We sometimes
write $\1$ for the terminal object and $\0$ for the initial object.

\setcounter{Exercise}{17}
\begin{Exercise}
    Verify the following claims. In each case, identify the canonical arrows.
    \begin{enumerate}
        \item In $\catname{Set}$, the empty set is an initial object while any
        one-element set $\{ \bullet \}$ is terminal.
        \item In $\catname{Pos}$, the poset $(\emptyset, \emptyset)$ is an
        initial object while $(\{ \bullet \}, \{ (\bullet, \bullet) \})$ is
        terminal.
        \item In $\catname{Top}$,  the space $(\emptyset, \{ \emptyset \})$ is
        an initial object while $(\{ \bullet \}, \{ \emptyset, \{ \bullet \} \})$
        is terminal.
        \item In $\catname{Vect}_k$, the one-element space $\{ 0 \}$ is both
        initial and terminal.
        \item In a poset, seen as a category, an initial object is a least
        element, while a terminal object is a greatest element.
    \end{enumerate}
\end{Exercise}
\begin{solution}
    \leavevmode
    \begin{enumerate}
       \item In $\catname{Set}$, for any set (object) $A$, the function
       $(\emptyset, A, \emptyset)$ is the unique function (arrow) from
       $\emptyset$ to $A$. Therefore, the empty set is (the) initial object in
       $\catname{Set}$. And, for every set $A$, the function $A \to
       \{ \bullet \}$ that maps every element of $A$ to $\bullet$ is the unique
       function from $A$ to $\{ \bullet \}$. This establishes that any
       one-element set is terminal in $\catname{Set}$.
       \item For any poset $(P, \le)$, there exists a unique (empty) monotone
       function $(\emptyset, \emptyset) \xrightarrow{(\emptyset, P, \emptyset)}
       (P, \le)$. Hence, the poset $(\emptyset, \emptyset)$ is an initial object
       in $\catname{Pos}$. And, for any poset $(P, \le)$, there exists a unique
       monotone function $(P, \le) \to (\{ \bullet \}, \{ (\bullet, \bullet) \})$,
       defined by $x \mapsto \bullet$ for all $x \in P$. Hence, $(\{ \bullet \},
       \{ (\bullet, \bullet) \})$ is terminal in $\catname{Pos}$.
       \item For any topological space $(X, T_X)$, the unique empty function
       \begin{center}
           $(\emptyset, \{ \emptyset \}) \xrightarrow{(\emptyset, X, \emptyset)}
           (X, T_X)$
       \end{center}
       is continuous, since for every open set $T \in T_X$, its preimage under
       the aforesaid function is the empty set, which is open. Hence,
       $(\emptyset, \{ \emptyset \})$ is initial in $\catname{Top}$.\\
       And, for any topological space $(X, T_X)$, the unique function
       $(X, T_X) \to (\{ \bullet \}, \{ \emptyset, \{ \bullet \} \})$, defined by
       $x \mapsto \bullet$ for all $x \in X$, is continuous, since the preimage
       of $\emptyset$ under the aforesaid function is $\emptyset$, which is open,
       and the preimage of $\{ \bullet \}$ is $X$, which is also open. Hence,
       $(\{ \bullet \}, \{ \emptyset, \{ \bullet \} \})$ is terminal in
       $\catname{Top}$.
       \item Assuming the ground field is $k$, for any vector space $V$, the
       unique linear map $\{ 0 \} \to V$, defined by $0 \mapsto 0_V$ is a unique
       arrow from $\{ 0 \}$ to $V$ in $\catname{Vect}_k$. Also, the unique linear
       map $V \to \{ 0\}$, defined by $v \mapsto 0$ for all $v \in V$, is a
       unique arrow from $V$ to $\{ 0 \}$ in $\catname{Vect}_k$. This shows that
       $\{ 0 \}$ is both initial and terminal in $\catname{Vect}_k$.
       \item In a poset $(P, \le)$, seen as a category, if $\bot$ is an initial
       object, then there exists a unique arrow $\bot \to p$ for all $p \in P$.
       This implies $\bot \le p$ for all $p \in P$, when seen as a set. Hence,
       an initial object in the category corresponding to $(P, \le)$ is a least
       element in $P$. Arguing similarly, we conclude that a terminal object in
       the category corresponding to $(P, \le)$ is a greatest element in $P$.
   \end{enumerate}
\end{solution}

\begin{Exercise}
    Identify the initial and terminal objects in $\catname{Rel}$.
\end{Exercise}
\begin{solution}
    In $\catname{Rel}$, the empty set $\emptyset$ is both the initial object and
    the terminal object. Indeed, for any set $A$, the empty relation $\emptyset$
    ($\subseteq \emptyset \times A$) is a unique relation from $\emptyset$ to
    $A$, and the empty relation $\emptyset$ ($\subseteq A \times \emptyset$) is
    also a unique relation from $A$ to $\emptyset$.
\end{solution}

\begin{Exercise}
    Suppose a monoid, viewed as a category, has either an initial or a terminal
    object. What must the monoid be?
\end{Exercise}
\begin{solution}
    The category corresponding to a monoid $(M, \cdot, 1_M)$ contains just a
    single object. If this object is initial, then all morphisms must be the
    identity morphism on this initial object, which implies $M = \{ 1_M \}$.
    The argument is similar if the aforesaid object is terminal, which would
    again imply $M = \{ 1_M \}$. Thus, in either case, the monoid must be the
    trivial monoid.
\end{solution}

\end{document}
